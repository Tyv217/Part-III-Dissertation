\documentclass[../main.tex]{subfiles}
\begin{document}
\chapter*{Abstract}

Regular supervised learning Machine Learning models learn to 
predict the labels of inputs.
Concept Bottleneck Models (CBMs) are ML models designed
to increase the interpretability of model predictions by decomposing a model into
two submodels, splitting the original process into predicting a set of human-interpretable
concepts / features present in the input, then predicting the label using these concepts.
Since these concepts are human-interpretable, it is much more easier to understand
the reasoning behind the predicted labels, thus mitigating some of the
potential dangerous downsides associated with using ML models as "black-box" models,
especially in fields where these predictions can have significant consequences to
human life, such as medicine, criminal justice, autonomous vehicles, etc. 

During inference time, professionals can intervene on CBMs by correcting
the predicted concepts leading to more accurate predicted labels. 
Due to the costs
associated with performing such an intervention, the question of what concepts
to intervene on in order to maximize the accuracy of the model becomes an important 
research question. This project focuses on answering this research question.
This project attempts to model the costs associated with using experts
to perform interventions as a budget, and thus the main research question 
of this project is
"How can we determine the concepts to intervene on for a given budget for 
a set of inputs and the corresponding model predictions?"

This project focuses on answering the above research questions. It first 
investigates the differences between greedy and non-greedy models, showing that non-greedy algorithms can outperform
its greedy counterparts. It then investigates
the performance of greedy models, building on top of existing methods by incorporating
 surrogate models to model the distribution
of concepts. The output of these surrogate models are then used by an ML
model that learns to predict the next concept to intervene on in each step. 
The project then investigates using Reinforcement Learning and these surrogate models to train a non-greedy model that 
learns to predict an entire sequence of interventions for given inputs and corresponding
CBM outputs.
Lastly, the project then investigates if it is necessary to train the prediction
model simultaneously with the CBMs.
\end{document}