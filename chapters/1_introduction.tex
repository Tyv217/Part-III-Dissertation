\documentclass[../main.tex]{subfiles}
\begin{document}
\chapter{Introduction}
\label{firstcontentpage}


Regular supervised Machine Learning (ML) models learn to 
predict the labels of inputs.
Concept Bottleneck Models~\cite{cbm} (CBMs) and Concept Embedding Models~\cite{cem} (CEMs) are 
ML models designed to increase the interpretability of model predictions by decomposing a model into
two submodels, splitting the original process into predicting a set of human-interpretable
concepts / features present in the input, then predicting the label using these concepts.
Since these concepts are human-interpretable, it becomes much easier to understand
the reasoning behind the predicted labels, mitigating some of the
potential dangerous downsides associated with using ML models as ``black-box'' models,
especially in fields where these predictions can have significant consequences to
human life, such as medicine, criminal justice, autonomous vehicles, etc. 

During inference time, professionals can intervene on CBMs and CEMs by correcting
the predicted concepts, leading to more accurate predicted labels. 
Due to the costs
associated with performing such an intervention, the question of finding what concepts
to intervene on in order to maximize the accuracy of the model becomes an important 
research question. Past methods have mainly focused on greedy approaches, which
includes using heuristic-based approaches such as intervening on concepts that minimize the
model's uncertainty~\cite{coop}, or ML-based greedy approaches that learn a intervention policy model
like in IntCEMs~\cite{intcem}. Additionally existing approaches either 
attempt to model the distribution of concepts and use that to perform interventions~\cite{energycem}, 
or directly learn a policy~\cite{intcem}, while limited
work has been done on combining the capabilities of models from these two approaches.
Compared to existing work, we investigate the possibility of learning non-greedy policies
where we model the problem of deciding what concepts to intervene on as a Reinforcement Learning~\cite{rl} 
problem. We model the costs
associated with interventions and budgets for interventions, which are constraints present in
 real life
applications, and set the main research question to be
determining the concepts to intervene on for different budgets for a CEM
to maximize its performance.

To answer the above question, we first investigate the differences between greedy and non-greedy policies,
proving that non-greedy algorithms can outperform
their greedy counterparts. We then build a Reinforcement Learning agent
that learns to select the next concept to intervene in order to maximize the model's performance for a given budget.
To guide the agent throughout the intervention process, we utilize surrogate models that model
the conditional distribution of concepts. A detailed description of these models are in Section~\ref{method:rlcem}.
These surrogate models guide the RL agent throughout the intervention
process by rewarding the agent for intervening on concepts that lead to higher probabilities of
intervened concepts, and provide auxiliary information about the unintervened concepts
 in order for the agent to make more informed decisions.
% The output of these surrogate models are then used by a Reinforcement Learning agent in order to
% perform sequential intervention decisions to maximize the final performance, which is trained
% in conjunction with a CBM in order to increase the sensitivity of the CBM to
% the interventions. 
We augment CEMs with this RL agent and a pretrained surrogate model to form RLCEM, 
and train the CEM and the RL agent simultaneously, which helps learn a non-greedy policy
and increase the sensitivity of the CEM to these interventions.

The results show that such RLCEMs
outperform existing models when intervened using the learnt policy, while achieving 
similar performance
under the absence of interventions. This project successfully demonstrates the how a non-greedy 
policy can be learnt using Reinforcement Learning and surrogate models that 
outperforms existing greedy policies for different budgets, showing that Reinforcement Learning
is a viable approach for learning more optimal intervention policies.

\end{document}