\documentclass[../main.tex]{subfiles}
\begin{document}
\chapter*{Abstract}




% open problem
Concept Bottleneck Models (CBMs) are designed for high interpretability by predicting 
human-interpretable concepts from input, 
then predicting labels from these concepts. During run-time, experts can intervene on the 
predicted concepts
by correcting them to improve the accuracy of the predicted label. Previous 
studies have shown that the choice of concepts to intervene on 
can have significant impacts to the accuracy of the CBM.
% and many efforts have been 
% made to develop methods to determine the best order of concepts to intervene on.
% why it is important and why we should care to solve this (motivation)
Since experts have limited time, it is important to develop methods that can 
determine the best order of concepts to intervene on to maximize the performance of the model 
while minimizing the costs that come with interventions. Given a budget
for interventions, we want to develop methods that achieve the best model performance
using the limited number of interventions allowed.
% your research question(s)
We investigate methods to learn a non-greedy policy that determines an optimal set of 
concepts to intervene on
for a given budget, with the goal of maximizing the label prediction accuracy of the underlying CBM.
% what other have done and how their work is deficient with regards to your question(s)
Past methods in developing intervention policies are limited in
that they only use greedy policies 
that attempt to maximize the accuracy of the CBM at each step.
They do not incorporate the notion of intervention budgets,
which we believe is an important issue to consider when using CBMs for real-life applications.
We hypothesize that non-greedy policies can find more optimal concepts to intervene on for a 
given budget,
as they maximize the final performance after a number of interventions rather than at each step.
% The key technical idea (not abstract but technical detail) and innovation that you did that enabled you to answer your question
To learn a non-greedy policy, we model the problem of determining what concepts to intervene on
as a Markov Decision Process
and apply Reinforcement Learning by training the model to maximize the final performance of the CBM
after a set number of interventions.
We also use a surrogate model to model the distribution of concepts to provide intermediate rewards and auxiliary information to the
Reinforcement Learning agent.
% How you did it (methodology)
Using a pre-trained surrogate model, a Reinforcement Learning agent 
is trained in conjunction with a CBM, 
which learns a policy for determining the choice of concepts
to intervene on for a given budget.
% What the results were
We first show that non-greedy policies can outperform 
greedy policies for different budgets.
We then show that we can learn a non-greedy policy using Reinforcement Learning which outperforms
existing greedy policies for different intervention budgets, 
while maintaining similar 
performance under the absence of interventions.
% What impact could your contributions have
In summary, We show that Reinforcement Learning serves as a viable approach to 
learning intervention policies more optimal than
existing appraoches for different budgets, 
opening the door to further research in non-greedy intervention policies.
% With the recent developments in Concept Bottleneck Models,
% finding optimal interventions for these models
% to maximize their performance under a given budget is an important issue, as it allows
% us to maximize the performance of these interpretable models while minimizing
% the costs related to expert interventions. This project investigates solving this problem
% using a new approach that combines Reinforcement Learning with surrogate models, with the goal
% of maximizing performance using intervention policies learnt by this new method.

% Compared to 
% previous approaches that only use greedy policies, 
% such as heuristic-based methods or learning a Machine Learning model for a greedy policy,
% the main idea of this project
% is to use Reinforcement Learning to learn a non-greedy policy to maximize the performance
% gain from interventions over a budget rather than at each step. This is done by modelling the problem 
% of deciding
% what concept to intervene on as a Reinforcement Learning problem to maximize the final performance,
% using surrogate models that model the distribution of concepts to provide intermediate rewards and 
% auxiliary information. 
% When used in conjunction
% with methods to increase sensitivity of models to interventions, this project successfully develops a novel
% RL-based method that outperforms existing greedy intervention policies for different budgets, while maintaining
% similar un-intervened performance. 

% This project proposes the idea of learning non-greedy
% intervention policies for Concept Bottleneck Models and incorporates budgets into the problem setting, 
% and shows that
% different budgets can have different optimal intervention trajectories where
% non-greedy policies outperform greedy policies.
% This project also successfully develops a novel approach to learning intervention policies using
% Reinforcement Learning and demonstrate that this approach generates 
% more optimal intervention policies
% that lead to better performance
% compared to existing greedy methods.
\end{document}