\csname documentclass\endcsname[../main.tex]{subfiles}
\begin{document}
\chapter{Summary and conclusions}

To summarize, we have presented a new CEM-based model,
which we name RLCEM, that utilizes Arbitrary Conditional Flow 
models and Reinforcement Learning to learn a non-greedy 
intervention policy. A variant of AC Flow model is used to 
learn the class-specific
conditional distribution between concepts, which is then used 
to train a Reinforcement Learning agent that learns
a non-greedy intervention policy. The AC Flow model 
provides intermediate rewards to the agent based on the
expected information gain to the target variable,
and the post-intervention Cross-Entropy loss is used as a final reward
to train the agent to make more optimal 
interventions that lead to 
higher label prediction accuracies. The sampled interventions
are also used to update the CEM to increase its sensitivity
to interventions, further boosting the performance of the learnt
intervention policy.

We have also evaluated RLCEM against existing greedy intervention
policies IntCEM, CooP, and a policy where we randomly select
concepts to intervene. We show that RLCEM was able to outperform 
these greedy intervention policies, learning a more optimal
non-greedy intervention policy. RLCEM achieves similar or 
better performance than the other policies while requiring
25\% less interventions, 
showing that it is able to learn a non-greedy policy that 
outperforms existing greedy policies in intervention
performance. At the same time,
we show that RLCEM achieves similar un-intervened performance
to existing models.

However, we also show that RLCEM
does not learn a more optimal intervention policy
in certain tasks due to sub-optimal AC Flow models.
We show that the performance of RLCEM is highly dependent on
the trained AC Flow model, which performance can vary. 
Hence, this method
is not very robust. Additionally, we show that 
RLCEM also suffers from other problems like 
high time complexity and concept leakage.

\section{Future Work}
Due to the limitations of RLCEM, there is room for future 
work to find solutions and improvements to these limitations.
For example, using a different approach to model
the distribution of concepts, such as 
a different latent distribution model for the AC Flow model,
or using a completely different surrogate models. Further 
research can investigate these different solutions to 
improve the robustness of the overall RLCEM 
by improving the robustness of the surrogate model itself.

Currently the 
AC Flow model uses an RNN based approach
to model the conditional distribution of concepts,
which takes $O(d)$ steps per output and is one of the reasons
for the high time complexity of our RLCEM approach.
 Future work can
look into using more efficient models to improve
the scalability of the AC Flow model, for example transformer-based
architectures that have replaced RNNs. However, the current
RNN-based approach uses a sequential approach
for sampling, which is not straightforward to translated
to a transformer-based architecture, and appropriate changes 
will have to be made.

Lastly, more research can be done on the effects
of concept leakage on modelling the distribution
of concepts in CEMs. Methods to combat the negative
effects of concept leakage, such as dynamically updating
the AC Flow model based on the distribution of concept 
embeddings may be worth exploring.

\end{document}