% Suggested LaTeX style template for Masters project report submitted at the
% Department of Computer Science and Technology
%
% Markus Kuhn, May 2022
% (borrowing elements from an earlier template by Steven Hand)

\documentclass[12pt,a4paper,twoside]{report}
% append option ",openright" after "twoside" if you prefer each chapter
% to start on a recto (odd-numbered) page in a double-sided printout

\usepackage[pdfborder={0 0 0}]{hyperref}  % turns references into hyperlinks
\usepackage[vmargin=20mm,hmargin=25mm]{geometry}  % adjust page margins
\usepackage{graphicx} % allows inclusion of PDF, PNG and JPG images
\usepackage{parskip}  % separate paragraphs with vertical space
                      % instead of indenting their first line
\usepackage{setspace} % for \onehalfspacing
\usepackage{refcount} % for counting pages
\usepackage{upquote}  % for correct quotation marks in verbatim text
\usepackage{verbatim}
\usepackage{docmute}   % only needed to allow inclusion of proposal.tex
\usepackage[utf8]{inputenc}
\usepackage{mathtools}
\usepackage{changepage}
\usepackage{url}
\usepackage{blindtext}
\usepackage{color}
\usepackage{amsfonts}
\usepackage{multirow}
\usepackage{multicol}
\usepackage{tabularx}
\usepackage{bm}
\usepackage{subfiles}
\usepackage{tikz}
\usepackage[edges]{forest}
\usepackage[font=small]{caption}
\usepackage{lscape}
\usepackage{xcolor}
\usepackage{tablefootnote}
\usepackage{tcolorbox}
\usepackage{dirtree}
\usepackage{hyperref}
\usepackage{listings}
\lstset{
  basicstyle=\fontfamily{cmtt}\selectfont,
  keywordstyle=\bfseries,
  mathescape,
  morecomment=[l][\color{gray}]{\#},
}
% \usepackage{algpseudocode}
\usepackage[ruled,vlined,linesnumbered]{algorithm2e}   
\usepackage{algorithmic}
%TC:group table 0 1
%TC:group tabular 1 1
%TC:group verbatim 0 1
%TC:group lstlisting 0 1
%TC:group algorithm 1 1
\newif\ifsubmission % Boolean flag for distinguishing submitted/final version

% Change the following lines to your own project title, name, college, course
\title{Ain't Nobody Got Time For That: Budget-aware Concept Intervention Policies}
\author{Thomas Yuan}
\date{May 2024}
\newcommand{\candidatenumber}{1234N}
\newcommand{\college}{Downing College}
\newcommand{\course}{Computer Science Tripos, Part III}
%\newcommand{\course}{Master of Philosophy in Advanced Computer Science}

% Select which version this is:
% For the (anonymous) submission (without your name or acknowledgements)
% uncomment the following line (or let the makefile do this for you)
%\submissiontrue
% For the final version (with your name) leave the above commented.

\begin{document}

%TC:ignore

\begin{sffamily} % use a sans-serif font for the pro-forma cover sheet

\begin{titlepage}
\makeatletter

% University logo with shield hanging in left margin
\hspace*{-14mm}\includegraphics[width=65mm]{logo-dcst-colour}

\ifsubmission

% submission proforma cover page for blind marking
\begin{Large}
\vspace{20mm}
Research project report title page

\vspace{35mm}
Candidate \candidatenumber

\vspace{42mm}
\textsl{``\@title''}

\end{Large}

\else

% regular cover page
\begin{center}
\Huge
\vspace{\fill}

\@title
\vspace{\fill}

\@author
\vspace{10mm}

\Large
\college
\vspace{\fill}

\@date
\vspace{\fill}

\end{center}

\fi

\vspace{\fill}
\begin{center}
Submitted in partial fulfillment of the requirements for the\\
\course
\end{center}

\makeatother
\end{titlepage}

\newpage

Total page count: \pageref{lastpage}

% calculate number of pages from
% \label{firstcontentpage} to \label{lastcontentpage} inclusive
\makeatletter
\@tempcnta=\getpagerefnumber{lastcontentpage}\relax%
\advance\@tempcnta by -\getpagerefnumber{firstcontentpage}%
\advance\@tempcnta by 1%
\xdef\contentpages{\the\@tempcnta}%
\makeatother

Main chapters (excluding front-matter, references and appendix):
\contentpages~pages
(pp~\pageref{firstcontentpage}--\pageref{lastcontentpage})

Main chapters word count: 467

Methodology used to generate that word count:

\begin{quote}
\begin{verbatim}
$ make wordcount
gs -q -dSAFER -sDEVICE=txtwrite -o - \
   -dFirstPage=6 -dLastPage=11 report-submission.pdf | \
egrep '[A-Za-z]{3}' | wc -w
467
\end{verbatim}
\end{quote}


\end{sffamily}

\vspace{\fill}
\onehalfspacing
\ifsubmission\else\makeatletter
\textbf{\Huge Declaration}
\vspace{40pt}

I, \@author\ of \college, being a candidate for the \course, hereby
declare that this report and the work described in it are my own work,
unaided except as may be specified below, and that the report does not
contain material that has already been used to any substantial extent
for a comparable purpose.

% Add here things like: Figure X is the work of Y, etc.

\bigskip 
\textbf{Signed: Thomas Yuan}

\bigskip
\textbf{Date: \today}
\vspace{\fill}
\makeatother\fi-

\chapter*{Abstract}

Regular supervised learning Machine Learning models learn to 
predict the labels of inputs.
Concept BottleNeck Models (CBMs) are ML models designed
to increase the interpretability of model predictions by decomposing a model into
two submodels, splitting the original process into predicting a set of human-interpretable
concepts / features present in the input, then predicting the label using these concepts.
Since these concepts are human-interpretable, it is much more easier to understand
the reasoning behind the predicted labels, thus mitigating some of the
potential dangerous downsides associated with using ML models as "black-box" models,
especially in fields where these predictions can have significant consequences to
human life, such as medicine, criminal justice, autonomous vehicles, etc. 

During inference time, professionals can intervene on CBMs by correcting
the predicted concepts leading to more accurate predicted labels. 
Due to the costs
associated with performing such an intervention, the question of what concepts
to intervene on in order to maximize the accuracy of the model becomes an important 
research question. This project focuses on answering this research question.
This project attempts to model the costs associated with using experts
to perform interventions as a budget, and thus the main research question 
of this project is
"How can we determine the concepts to intervene on for a given budget for 
a set of inputs and the corresponding model predictions?"

This project focuses on answering the above research questions. It first 
investigates the differences between greedy and non-greedy models, showing that non-greedy algorithms can outperform
its greedy counterparts. It then investigates
the performance of greedy models, building on top of existing methods by incorporating
 surrogate models to model the distribution
of concepts. The output of these surrogate models are then used by an ML
model that learns to predict the next concept to intervene on in each step. 
The project then investigates using Reinforcement Learning and these surrogate models to train a non-greedy model that 
learns to predict an entire sequence of interventions for given inputs and corresponding
CBM outputs.
Lastly, the project then investigates if it is necessary to train the prediction
model simultaneously with the CBMs.

\ifsubmission\else
% not included in submission for blind marking:

\chapter*{Acknowledgements}

This project would not have been possible without the wonderful
support of my lovely supervisors Mateo Espinosa Zarlenga, Dr Mateja Jamnik and Dr. Zohreh Shams. I would also like to 
thank my friends and family for their support.

\fi
\cleardoublepage % preserve page numbers after missing acknowledgements

\tableofcontents
\listoffigures
\listoftables


%TC:endignore

\pagestyle{headings}
\subfile{chapters/1_introduction.tex}
\subfile{chapters/2_background.tex}
\subfile{chapters/3_related_work.tex}
\subfile{chapters/4_implementation.tex}
\subfile{chapters/5_evaluation.tex}
\subfile{chapters/6_conclusion.tex}
\label{lastcontentpage}

%%%%%%%%%%%%%%%%%%%%%%%%%%%%%%%%%%%%%%%%%%%%%%%%%%%%%%%%%%%%%%%%%%%%%
% the bibliography

%TC:ignore
\addcontentsline{toc}{chapter}{Bibliography}
\bibliographystyle{plain}
\bibliography{refs}

%%%%%%%%%%%%%%%%%%%%%%%%%%%%%%%%%%%%%%%%%%%%%%%%%%%%%%%%%%%%%%%%%%%%%
% the appendices
\subfile{chapters/7_appendix.tex}

\label{lastpage}

%TC:endignore
\end{document}
